\documentclass{article}
\usepackage{graphicx} % Required for inserting images
\setlength{\textwidth}{7.0in}
\setlength{\oddsidemargin}{-0.35in}
\setlength{\topmargin}{-0.5in}
\setlength{\textheight}{9.0in}
\setlength{\parindent}{0.3in}

\begin{document}

\begin{flushleft}
\textbf{Author: Tanishq Harit / tharit@stevens.edu / harit.tani@gmail.com}
\end{flushleft}
\begin{center}
\textbf{\Large Probability for Machine Learning} \\
\end{center}

\section*{Probability Terminologies:}
\begin{itemize}
    \item \textbf{Random Experiment}: An experiment is called random experiment if it satisfies following conditions:
    \begin{itemize}
        \item It has more than one possible outcome.
        \item It is not possible to predict outcome in advance.
        \item Example: Tossing a coin (possibilities - heads or tails)
    \end{itemize}
    \item \textbf{Trial and Outcome}: Trial is single execution of a random experiment. Each trial produces an Outcome. In tossing a coin, when we toss the coin it is called trial and whether the coin is heads or tails is its outcome.
    \item \textbf{Sample Space}: Sample space of a random experiment is the set of all possible outcomes that can occur. Generally, one random experiment will have one \textbf{set} of sample space.
    \begin{itemize}
        \item Again in tossing a coin example, we have sample space of \{H,T\} or \{Heads, Tails\}.
        \item Another example is rolling a dice, which will have a sample space of \{1, 2, 3, 4, 5, 6\}.
    \end{itemize}
    \item \textbf{Event}: Event is specific set of outcomes from a random experiment or process. Essentially, it's a subset of sample space. An event can include a single outcome, or it can include multiple outcomes. One random experiment can have multiple events.
    \begin{itemize}
        \item In rolling a dice example, if we want to measure the probability of getting a number less than 3 then our event is "number less than 3" and this would be \{1, 2\}. As we can see, this is still subset of sample space \{1, 2, 3, 4, 5, 6\}.
        \item We can have multiple events in rolling the dice example. Getting an odd number \{1, 3, 5\} or getting an even number \{2, 4, 6\}. Again, both of these events are subset of sample space \{1, 2, 3, 4, 5, 6\}.
    \end{itemize}
\end{itemize}

\section*{Examples:}
Now, we will take some examples and for every example, we will understand their terminologies.
\begin{itemize}
    \item Example 1 \textbf{Rolling a Dice}: 
    \begin{itemize}
        \item Random Experiment: Rolling the dice.
        \item Trial: Rolling the die once. 
        \item Outcome: Suppose, when we rolled it, we got outcome of \{3\}. Maybe, if we roll again, we get outcome of \{2\} or \{5\} or any other number.
        \item Sample Space: \{1, 2, 3, 4, 5, 6\}
        \item Event: Could be anything, depends on the question or a person.
    \end{itemize}
    \item Example 2 \textbf{Tossing a Coin Twice}: Tossing the coin once and then tossing it again.
    \begin{itemize}
        \item Random Experiment: Tossing the coin twice.
        \item Trial: Tossing the coin twice (once).
        \item Outcome: Let's say, when first time we tossed it we got Heads and when we toss it for the second time, we get Tails. Som outcome is \{Heads, Tails\}.
        \item Sample Space: \{\{H,H\}, \{H,T\}, \{T,H\}, \{T,T\}\}
        \item Event: Again, event could be anything. Example: getting 2 heads then outcome would be \{H,H\} or getting at-least one head then outcome would be \{\{H,H\}, \{H,T\}, \{T,H\}\}
    \end{itemize}
    \newpage
    \item Example 3 \textbf{Titanic Dataset Example}: There are 891 passengers, randomly picking out 1 passenger than calculating it's P-class.
    \begin{itemize}
        \item Random Experiment: Randomly drawing out a passenger.
        \item Trial: Randomly drawing out the passenger and finding it's p-class.
        \item Outcome: Let's say, there are 3 p-class \{1, 2, 3\} then outcome can be \{1\} or \{2\} or maybe \{3\}.
        \item Sample Space: Take any passenger, it's p-class would be in \{1, 2, 3\}. So, sample space is \{1, 2, 3\}.
        \item Event: Again, it could be anything. Example: .A passenger is from p-class \{3\} or a passenger is not from p-class \{2\}.
    \end{itemize}
\end{itemize}

\section*{Types of Events:}
\begin{itemize}
    \item \textbf{Simple Event}: Event that consists os exactly one outcome. Also known as elementary event.
    \begin{itemize}
        \item When rolling a die, getting a 6 is a simple but getting odd numbers is not a simple event.
    \end{itemize}
    \item \textbf{Compound Event}: Consists of two or more simple events. 
    \begin{itemize}
        \item When rolling a die, event "rolling an odd number" is a compound event because it consists of 3 simple events: rolling a 1, rolling a 3 and rolling a 5.
    \end{itemize}
    \item \textbf{Independent Event}: 
\end{itemize}

\end{document}